%pdflatex README.tex
\documentclass{article}
\usepackage{hyperref}
\title{HACMS CMU Path Planner, Navigaiton, and Monitors}
\author{Jason Larkin}
\date{03/09/2015}
\begin{document}
\maketitle

\section{Introduciton}

The Path Planner, Navigation, and Monitors are discussed in the context of video 

\href{https://www.dropbox.com/s/jlr5wiiuo6to4hc/landshark_2dnav_demo.ogv?dl=0}{landshark 2dnav demo.ogv}

\section{Path Planner and Navigation}
\begin{itemize}
\item 0:02:
The map input format is latitude, longitude, and obstacle "size" (in meters). 
The obstacles are defined in a simple text file (e.g., obstacles.map in video). ARAVIND: what should I say about how the file is specified and loaded by OCU? :ARAVIND The 
goal is also defined in lat/lon format and is specified by the OCU. ARAVIND: How is goal defined by OCU? :ARAVIND
\item 0:04:
While the map's input format is in lat/lon, the robot position, obstacle positions, 
etc. are represented on a map using a local UTM meters frame.  By default, the 
position of the robot when the navigation system is launched is considered 
the origin (0,0). The origin can be set to the robot's current position after initial launch if desired. ARAVIND: do you want to add an origin reset to OCU GUI? :ARAVIND

The obstacle "size" 
is a square region (red transparent squares) defined around the point obstacles 
(solid red circles). The obstacle "size" 
can be set to a value so that the robot's local path planning controller 
can avoid the dynamic window region (transparent red circles) defined by 
the dynamic window monitor (see Section \ref{Section_Monitors}). 
\item 0:14-0:16:
The Path Planner and Navigation system are requested.  This is done with the joystick on the physical Landshark. ARAVIND: which button(s)? :ARAVIND
\item 0:17:
The navigation plan is made and the navigation begins (Status: engaged). 
\item 0:30:
The current navigation plan is cancelled (Status: off).
\item 0:35:
A new navigation plan is requested and navigation continues. 
\end{itemize}

\section{Monitors}\label{Section_Monitors}
This section of the video demonstrates the dynamic window monitor 
behavior.
\begin{itemize}
\item 0:44:
Navigation is cancelled. 
\item 0:48:
Naviation switches to joystick control and the robot is driven deliberatly 
at the obstacle and its dynamic window. Note: the window is not represented 
dynamically in the video. 
\item 0:53:
Dynamic window monitor calls ESTOP when robot enters the obstacle's dynamic window 
region. On the physical Landshark the ESTOP will result in robot braking to a stop.  
Joystick control can be... ARAVIND: How is ESTOP cancelled and joystick control 
restored? :ARAVIND
\end{itemize}
\end{document}

